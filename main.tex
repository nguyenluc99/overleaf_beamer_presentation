\documentclass[handout]{beamer}
\usepackage[utf8]{inputenc}
% \usepackage[utf8]{vietnam}
\usepackage{amsmath}
\usepackage{hyperref}
\usepackage{xcolor}
\usepackage{multicol}
\usepackage{listings}
\usepackage{verbatim}
\usepackage{multirow}
\usepackage{fontawesome5}
\usepackage{wrapfig}
\usepackage{tikz}
\usepackage{graphicx}
\graphicspath{{./Images/}}
\newcommand\calculategraphicstargetheight[1]{%
     \setlength\graphht{\textheight 
                       -\parskip
                       -\abovecaptionskip -\belowcaptionskip
                       -(12pt * #1) % assuming baselineskip of 12pt in caption
                       }}
                       
\usepackage{color}
\usepackage{enumerate}
\usepackage{multirow}
\mode<presentation> {

% The Beamer class comes with a number of default slide themes
% which change the colors and layouts of slides. Below this is a list
% of all the themes, uncomment each in turn to see what they look like.

\lstset{
  basicstyle=\ttfamily,
  % basicstyle={\small\ttfamily},
  columns=fullflexible,
  % columns=flexible,
  frame=single,
  breaklines=true,
  postbreak=\mbox{\textcolor{red}{$\hookrightarrow$}\space},
  language=Java,
  aboveskip=3mm,
  belowskip=3mm,
  showstringspaces=false,
  numbers=none,
  numberstyle=\tiny\color{gray},
  keywordstyle=\color{blue},
  commentstyle=\color{dkgreen},
  stringstyle=\color{mauve},
  breakatwhitespace=true,
  tabsize=3
}


% \usetheme{default}
% \usetheme{AnnArbor}
% \usetheme{Antibes}
% \usetheme{Bergen}
% \usetheme{Berkeley}
% \usetheme{Berlin}
% \usetheme{Boadilla}
% \usetheme{CambridgeUS}
% \usetheme{Copenhagen}
% \usetheme{Darmstadt}
% \usetheme{Dresden}
% \usetheme{Frankfurt}
% \usetheme{Goettingen}
% \usetheme{Hannover} %%%
\usetheme{Ilmenau}
% \usetheme{JuanLesPins}
% \usetheme{Luebeck}
% \usetheme{Madrid}
% \usetheme{Malmoe}
% \usetheme{Marburg}
% \usetheme{Montpellier}
% \usetheme{PaloAlto}
% \usetheme{Pittsburgh}
% \usetheme{Rochester}
% \usetheme{Singapore}
% \usetheme{Szeged}
% \usetheme{Warsaw}

% As well as themes, the Beamer class has a number of color themes
% for any slide theme. Uncomment each of these in turn to see how it
% changes the colors of your current slide theme.

% \usecolortheme{albatross}
\usecolortheme{beaver} %
% \usecolortheme{beetle}
% \usecolortheme{crane}
% \usecolortheme{dolphin}
% \usecolortheme{dove}
% \usecolortheme{fly}
% \usecolortheme{lily}
% \usecolortheme{orchid}
% \usecolortheme{rose} % here
%\usecolortheme{seagull}
% \usecolortheme{seahorse}
% \usecolortheme{whale}
%\usecolortheme{wolverine}

%\setbeamertemplate{footline} % To remove the footer line in all slides uncomment this line
\setbeamertemplate{footline}[page number] % To replace the footer line in all slides with a simple slide count uncomment this line

%\setbeamertemplate{navigation symbols}{} % To remove the navigation symbols from the bottom of all slides uncomment this line
}




\title{\LaTeX\ \ Introduction}
\author{Luc Nguyen}
\date{June 2021}
\begin{document}

% \begin{frame}{Contents}
%     \titlepage
% \end{frame}
\maketitle

\begin{frame}{Outline}
    \tableofcontents
\end{frame}


\section{Once upon a time}

\begin{frame}{Once upon a time}
The birth of \LaTeX:
    \begin{itemize}
        \item Early version: \TeX, developed by Donald Knuth since 1978, originally is a ``computer language'' designed for use in typesetting, in particular, math and other technical expression.
        \pause
        \item \LaTeX (since 1980s by Leslie Lamport), is a document preparation system for high-quality typesetting. \\
        Microsoft Word first released on October 25, 1983.
        \pause
        \item KaTeX (\href{https://katex.org/}{katex.org}): a fast, self-contained JavaScript library that makes it easy to render TeX. {\Huge \color{red}\textbf{?????????}}
    \end{itemize}
    
\end{frame}




% \input{section2.tex}
 


% { % all template changes are local to this group.
%     \setbeamertemplate{navigation symbols}{}
%     \begin{frame}<article:0>[plain]
%         \begin{tikzpicture}[remember picture,overlay]
%             \node[at=(current page.center)] {
%                 \includegraphics[keepaspectratio,
%                                  width=\paperwidth,
%                                  height=\paperheight]{NVL_6602_1-3.jpg}
%             };
%         \end{tikzpicture}
%      \end{frame}
% }
% 

% \begin{frame}{test}
% \end{frame}
\end{document}

